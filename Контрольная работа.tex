\documentclass{article}

\usepackage[utf8]{inputenc}
\usepackage[russian]{babel}
\usepackage{listings}


\title{Технологии программирования \large \\ Контрольная работа\\ \bigskip{}  \huge{Вариант 11} }
\author{Роман Чулков}

\begin{document}
\maketitle
\section{Описание работы программы}

Задачей является описание класса работника, содержащего название должности, имя, фамилию и оклад. Также нам наобходимо предоставить интерфейс для работы с классом, т.к. в процессе использования класса в программе нам потребуется читать значения полей, изменять их, а также проводить анализ над группой таких объектов.

Наш класс \emph{Employee} определен в заголовочном файле \emph{employee.hpp}, который мы сможем подключать ко всем модулям программы, которым этот класс потребуется. Определение некоторых методов вынесено в отдельный модуль \emph{employee.cpp} ради целей ускорения компиляции и облегчение чтения кода класса.

В модуле \emph{main.cpp} определена главная функция \emph{main()}, а также несколько вспомогательных функций для выполнения задач задания массива работников, вывода информации о них и вывода информации о самом высокооплачиваемом работнике. Все наши вспомогательные функции принимают единственным аргументом \emph{Employee*} --- то есть, указатель на начало массива работников --- ради целей быстрого доступа к объектам и возможности изменять их поля, сохраняя результат после возврата из этих функций.

Внутри функции \emph{main()} происходит последовательное выполнение инструкций, требуемых в задании. Комментарии к ним описаны внутри файла с кодом программы. В конце функции \emph{main()} мы вызываем инструкцию \emph{return 0;} --- сигнал к завершению работы программы и возврат значения \emph{0} в качестве кода возврата, как факт того, что программа завершила свою работу без ошибок.

\pagebreak
\section{Ответы на вопросы}
\begin{enumerate}


	\item Что такое класс в объектно-ориентированном программировании?
	
		\begin{itemize}
		
			\item[-] Класс --- это описание типа объекта, взятого из предметной области, содержащее в себе описание свойств (полей) объекта и интерфейса для работы с ними. 
	
		\end{itemize}
	
	\item Какую структуру имеет модуль в С++?

		\begin{itemize}
		
			\item[-] Типичная структура модуля трансляции: подключение заголовочных файлов с помощью деректив препроцессора, объявление или определение функций, методов классов и глобальных переменных, определение функции \emph{main()}, если модуль является главным.
	
		\end{itemize}
	
	\item Какими средствами осуществляется консольный ввод данных в языке Си, С++?

		\begin{itemize}
		
			\item[-] В языке C консольный ввод осуществляется с помощью функции \emph{scanf}, определенной в заголовочном файле \emph{stdio.h}
			\item[-] В языке C++ консольный ввод принято осуществлять с помощью потока ввода \emph{std::cin}, определенного в файле \emph{iostream}
	
		\end{itemize}
	
	\item Какие свойства (принципы) объектно-ориентированного программирования вы знаете?

		\begin{itemize}
		
			\item[-] Инкапсуляция --- сокрытие внутренней реализации классов от внешнего пользователя и предоставление интерфейса для работы с объектом данного класса.
			\item[-] Наследование --- возможность использования полей и методов одного класса (\emph{родителя}) внутри другого класса (\emph{наследника}).
			\item[-] Полиморфизм --- возможность для работы с различными сущностями через одинаковый для всех интерфейс.
	
		\end{itemize}
	
	\item Сконструируйте простейший класс с конструктором по умолчанию и конструктором с параметрами. Покажите, как с помощью этих конструкторов можно создать объекты.

		\begin{itemize}
		
			\item[-] Определим простейший класс, представляющий собой элемент односвязного списка, хранящего целочисленное значение. Полям класса явно укажем \emph{private} спецификатор доступа.
				
				\begin{lstlisting}[language=C++]
    class Node {		
    private:
        int value_ = 0;
        Node* prev_ = nullptr;
    };
				\end{lstlisting}
	
			\item[-] Добавим в класс конструктор по умолчанию и конструктор с параметрами в \emph{public} области.

				\begin{lstlisting}[language=C++]

    class Node {	
    public:
        Node() = default;
        Node(int value, Node* prev = nullptr):
             value_(value),
             prev_(prev)
        {}

    private:
        int value_ = 0;
        Node* prev_ = nullptr;
    };
				\end{lstlisting}
	
			\item[-] Теперь можем создать объекты нашего класса.

				\begin{lstlisting}[language=C++]
				
    int main() {
        Node *first_number = new Node();
        Node *second_number = new Node(2, first_number);
        Node *third_number = new Node(15, second_number);

        return 0;
    }
				\end{lstlisting}
	
		\end{itemize}
	
\end{enumerate}
\end{document}
